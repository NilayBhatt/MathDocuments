\documentclass[14pt, fullpage]{article}
\usepackage{amssymb,latexsym,amsmath,amscd,epsfig,amsthm,graphicx,verbatim, extsizes}

\newcommand*{\QEDA}{\hfill\ensuremath{\blacksquare}}%
\pagestyle{empty}

\input epsf
\newdimen\epsfxsize

\parindent=0pt
%\setlength{\evensidemargin}{-2.0cm}
%\setlength{\oddsidemargin}{1.5cm}
\setlength{\topmargin}{-1.5cm}
%\setlength{\baselineskip}{20pt}
%\setlength{\textwidth}{19cm}
%\setlength{\textheight}{23cm}

\newcommand{\ds}{\displaystyle}
\newcommand{\un}{\underline}
\newcommand{\R}{\mathbb R}
\newcommand{\Z}{\mathbb Z}
\newcommand{\N}{\mathbb N}


\begin{document}
\begin{center}
		
{\bf MATH 450 Seminar in Proof}
 \\
\end{center}
	Prove that $f:A \rightarrow P(A)$ is not onto (Where $P(A)$ is the power set of $A$). Hint: consider the set $\{a \in A: a \notin f(a)\}$
\begin{proof}
	Let us proceed by contradiction. Let $f:A \rightarrow P(A)$ be onto. Then $f[A] = P(A)$. Let $B = \{a \in A: a \notin f(a)\}$. Thus $B \subset A$ and also $B \in f[A]$. Since $f$ is onto, we know that $B$ has a pre-image in $A$. Let $f(a) = B$, where $a$ be an element in $A$. There are two possibilities $a \in B$ or $a \notin B$. If $a \in B$ then by definition $a \notin f(a)$ but then since $f(a) = B$ there is a contradiction. If $a \notin B$ then by definition $a \in f(a)$ but then since $f(a) = B$ then again there is a contradiction. Thus $f$ is not onto.
\end{proof}
\end{document}