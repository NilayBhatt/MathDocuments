\documentclass[14pt]{article}
\usepackage{amssymb,latexsym,amsmath,amscd,epsfig,amsthm,graphicx,verbatim, extsizes}

\newcommand*{\QEDA}{\hfill\ensuremath{\blacksquare}}%
\pagestyle{empty}

\input epsf
\newdimen\epsfxsize

\parindent=0pt
%\setlength{\evensidemargin}{-2.0cm}
%\setlength{\oddsidemargin}{1.5cm}
\setlength{\topmargin}{-1.5cm}
%\setlength{\baselineskip}{20pt}
%\setlength{\textwidth}{19cm}
%\setlength{\textheight}{23cm}

\newcommand{\ds}{\displaystyle}
\newcommand{\un}{\underline}
\newcommand{\R}{\mathbb R}
\newcommand{\Z}{\mathbb Z}
\newcommand{\N}{\mathbb N}


\begin{document}
\begin{center}
		
{\bf MATH 450 Seminar in Proof}
 \\
\end{center}
	Prove that the set of odd integers are countable.
\begin{proof}
	 We know that a set $X$ is countable if there exists a bijection in $f: \N \rightarrow X$. We will show that $f:\N \rightarrow 2\Z - 1$ this is a well defined function and is a bijection. 
	 Let \[
f(n) =
\begin{cases}
 n & \text{if n is odd} \\
 1 - n & \text{if n is even}
\end{cases}
\]
We first show that the function is well defined meaning $f(n) \in 2\Z - 1 $ for every $n \in N$. If $n$ is even then $n = 2k$, $k > 0 \in \Z$. Thus $f(n) = 1 - 2k$. Since $2k > 0$ and is even, $1 - 2k < 0$ and is odd, thus $f(n) \in 2\Z - 1$. If $n$ is odd then $n = 2k - 1$, where $k > 0 \in \Z$. Then $f(n) = n = 2k - 1 \in \Z$. Thus our function is well defined. Now we will prove that it is bijective.

\textbf{One-to-One: \\}
Let $f(a) = f(b)$ where $a,b \in \N$. Then since, $f(a)= f(b)$ we know that either $f(a) = a$ and $f(b) = b$ or $f(a) = 1 - a$ and $f(b) = 1 - b$. In both the cases we get $a=b$ if $f(a) = f(b)$. Thus the function is one-to-one.\\ \textbf{Onto: \\}
Let $y \in \Z$. If $y > 0$ then we have $f(n) = y \rightarrow n = y$. Since $y > 0 \rightarrow n > 0 \in \N$. Thus $y = n \in \N$. If $y < 0$ then we have $ y = 1 -n$. Solving for $n$ we get $n = 1 - y$. Since $y <0$, $-y > 0$ and thus $-y + 1 > 0$. Thus $-y+1 = n \in \N$. Hence, the function is onto.

Thus the $f:\N \rightarrow 2\Z - 1$ with \[
f(n) =
\begin{cases}
 n & \text{if n is odd} \\
 1 - n & \text{if n is even}
\end{cases}
\] is well defined and is bijective. Thus the set of odd integers are countable.
\end{proof}
\end{document}