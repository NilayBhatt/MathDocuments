\documentclass[14pt]{article}
\usepackage{amssymb,latexsym,amsmath,amscd,epsfig,amsthm,graphicx,verbatim, extsizes}

\newcommand*{\QEDA}{\hfill\ensuremath{\blacksquare}}%
\pagestyle{empty}

\input epsf
\newdimen\epsfxsize

\parindent=0pt
%\setlength{\evensidemargin}{-2.0cm}
%\setlength{\oddsidemargin}{1.5cm}
\setlength{\topmargin}{-1.5cm}
%\setlength{\baselineskip}{20pt}
%\setlength{\textwidth}{19cm}
%\setlength{\textheight}{23cm}

\newcommand{\ds}{\displaystyle}
\newcommand{\un}{\underline}
\newcommand{\R}{\mathbb R}
\newcommand{\Z}{\mathbb Z}
\newcommand{\N}{\mathbb N}


\begin{document}
\begin{center}
		
{\bf MATH 450 Seminar in Proof}
 \\
\end{center}
	Prove that the set of odd negative integers is countable.\\
\begin{proof}
	 We know that a set $X$ is countable if there exists a bijection in $f: \N \rightarrow X$. Let $X$ be the set of negative odd integers.
	 Let $f:\N \rightarrow X $ such that $f(n) = 1-2n$. \\
We first show that the function has a valid co-domain. For every $n \in \N$ we have $f(n) = 1- 2n$. We know that $n >0$ then $-2n < 0$ and is even. Thus, $1-2n$ is a negative odd integer. Now we will prove that $f(n)$ is bijective.

\textbf{One-to-One: \\}
Let $f(a) = f(b)$ where $a,b \in \N$. Then since, $f(a)= f(b)$ we know that $1-2a = 1-2b$. Thus $a=b$. Therefore $f$ is one-to-one.\\
\textbf{Onto: \\}
Let $y \in -2\Z -1$ then $y < 0$. Observe that $f\left(\dfrac{1-y}{2}\right) = 1-2\left(\dfrac{1-y}{2}\right)= y$. Since $y \leq -1$ then we know that $-y \geq 1$, which means $ 1- y \geq 2$ and thus, $\dfrac{1-y}{2} \geq 1$. Also, since $y$ is a negative odd integer, $1-y$ is a positive even integer and thus $\dfrac{1-y}{2} \in \N$. Hence, the function is onto.

Thus $f:\N \rightarrow X$ with $f(n) = 1-2n$ is bijective. Thus the set of odd negative integers are countable.
\end{proof}
\end{document}