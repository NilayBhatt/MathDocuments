\documentclass[12pt, fullpage]{article}
\usepackage{amssymb,latexsym,amsmath,amscd,epsfig,amsthm,graphicx}

\newcommand*{\QEDA}{\hfill\ensuremath{\blacksquare}}%
\pagestyle{empty}

\input epsf
\newdimen\epsfxsize

\parindent=0pt
\setlength{\evensidemargin}{0.0cm}
\setlength{\oddsidemargin}{-1.7cm}
\setlength{\topmargin}{-3.2cm}
%\setlength{\baselineskip}{20pt}
\setlength{\textwidth}{19cm}
\setlength{\textheight}{23cm}

\newcommand{\ds}{\displaystyle}
\newcommand{\un}{\underline}
\newcommand{\R}{\mathbb R}
\newcommand{\Z}{\mathbb Z}


\begin{document}
\begin{flushleft}
\textbf{Nilay Bhatt March. 18 2017}		
\end{flushleft}
\begin{center}		
{\bf MATH 450 Seminar in Proof}
 \\
\end{center}
\textbf{Prove}: If $f$ is a function defined as $f: A \rightarrow B$ such that $A$ and $B$ are finite and $\vert A \vert = \vert B \vert$, then $f$ is one-to-one if only if it is onto.

\begin{proof}
$\\ \Longleftarrow$ Let $f: A \rightarrow B$ be a function such that $\vert A \vert = \vert B \vert$ and $f$ is onto ($f[A] = B$). Let $f$ be not one-to-one. Then there exists $a_1,a_2 \in A, a_1 \neq a_2$ such that $f(a_1) = f(a_2) = b$, $b \in B$. This means that $|f[A]| = |f(A-\{a_2\}|$. Also, by our assumption since $\vert A \vert = \vert B \vert$ this implies that $|f[A]| = |A|$, hence there is a contradiction $\rightarrow\leftarrow$, we will have an element in $B$ which does not have a pre-image in $A$ making it not onto. Thus $f$ is one-to-one.\\


$\Longrightarrow$  Let $f: A \rightarrow B$ be a function such that $\vert A \vert = \vert B \vert$ and $f$ is one-to-one. Then from the definition of one-to-one, if $f(a_1) = f(a_2)$ then $a_1 = a_2$. Also note that since $f$ is one-to-one $|A| =\left|f[A]\right|$. Let $f$ be not onto. Then there exist a $b \in B$ such that there does not exist any $a \in A$ where $f(a) = b$. This is a contradiction because, according to our assumption $ |A| = |B|$ implies $|f[A]| = |B|$ but then if $f$ is not onto it implies that there exists more elements in $B$ than there are in.  This is where our contradiction lies. Therefore $f$ has to be onto.
\end{proof}

\end{document}