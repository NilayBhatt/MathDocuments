\documentclass[12pt, fullpage]{article}
\usepackage{amssymb,latexsym,amsmath,amscd,epsfig,amsthm,graphicx}
\newcommand*{\QEDA}{\hfill\ensuremath{\blacksquare}}%
\pagestyle{empty}

\input epsf
\newdimen\epsfxsize

\parindent=0pt
\setlength{\evensidemargin}{0.0cm}
\setlength{\oddsidemargin}{-1.7cm}
\setlength{\topmargin}{-3.2cm}
%\setlength{\baselineskip}{20pt}
\setlength{\textwidth}{19cm}
\setlength{\textheight}{23cm}

\newcommand{\ds}{\displaystyle}
\newcommand{\un}{\underline}
\newcommand{\R}{\mathbb R}


\begin{document}

\begin{flushleft}
\textbf{Nilay Bhatt Feb. 20 2017}		
\end{flushleft}

\begin{center}
{\bf MATH 450 Seminar in Proof}
\end{center}

\begin{center}
\textbf{0.$\bar{9}$ = 1}
\end{center}

\begin{proof}
We know that:
\begin{align*}
0.9 &= \dfrac{9}{10} + \dfrac{9}{100} + \dfrac{9}{1000} + \dfrac{9}{100000} + ....\\
&= 9(\dfrac{1}{10} + \dfrac{1}{100} + \dfrac{1}{1000} + \dfrac{1}{100000} + ....)\\
&= 9 \sum_{n=1}^{\infty}\dfrac{1}{10^n} \\
&= 9 (\dfrac{\dfrac{1}{10}}{1-\dfrac{1}{10}})\\
&= 9 (\dfrac{\dfrac{1}{10}}{\dfrac{9}{10}})
&= 1
\end{align*}
Thus proved.

\end{proof}

\end{document}