\documentclass[12pt, fullpage]{article}
\usepackage{amssymb,latexsym,amsmath,amscd,epsfig,amsthm,graphicx}
\newcommand*{\QEDA}{\hfill\ensuremath{\blacksquare}}%
\pagestyle{empty}

\input epsf
\newdimen\epsfxsize

\parindent=0pt
\setlength{\evensidemargin}{0.0cm}
\setlength{\oddsidemargin}{-1.7cm}
\setlength{\topmargin}{-3.2cm}
%\setlength{\baselineskip}{20pt}
\setlength{\textwidth}{19cm}
\setlength{\textheight}{23cm}

\newcommand{\ds}{\displaystyle}
\newcommand{\un}{\underline}
\newcommand{\R}{\mathbb R}


\begin{document}

\begin{flushleft}
\textbf{Nilay Bhatt Feb. 19 2017}		
\end{flushleft}

\begin{center}
	\textbf{Divisibility of Three}
\end{center}

\begin{center}
		
{\bf MATH 450 Seminar in Proof}
 \\
\end{center}

\begin{center}
\textbf{A number is divisible by three if and only if the sum of the digits is divisible by 3.}
\end{center}

\begin{proof}
$\Rightarrow$ Let $\beta$ be a whole number divisible by $3$. We can then define $\beta$ as:
\begin{align*}
\beta &= \beta_0*10^0 + \beta_1 * 10^1 + \beta_2 * 10^2 + ... + \beta_i * 10^i\\
(\beta)\mod 3 &= ( \beta_0*10^0 + \beta_1 * 10^1 + \beta_2 * 10^2 + ... + \beta_i * 10^i ) \mod 3\\
0 &= ( \beta_0*10^0 + \beta_1 * 10^1 + \beta_2 * 10^2 + ... + \beta_i * 10^i ) \mod 3  \\
&= ( (\beta_0*10^0) \mod 3 + (\beta_1 * 10^1) \mod 3 + (\beta_2 * 10^2) \mod 3 + ... + (\beta_i * 10^i) \mod 3 )\mod 3 \\
&= ( ((\beta_0)\mod 3*(10^0)\mod3) \mod 3 + ((\beta_1)\mod 3 * (10^1)\mod 3) \mod 3 \\& +((\beta_2)\mod 3 * (10^2))\mod 3) \mod 3 + ... + ((\beta_i)\mod 3 * (10^i)\mod 3) \mod 3 )\mod 3 \\
&=( ((\beta_0)\mod 3*1) \mod 3 + ((\beta_1)\mod 3 * 1) \mod 3 +((\beta_2)\mod 3 * 1)\mod 3) \\& + \mod 3 + ... + ((\beta_i)\mod 3 * 1) \mod 3 )\mod 3\\
&= ( (\beta_0) \mod 3 + (\beta_1)\mod 3+(\beta_2)\mod 3 + ... + (\beta_i)\mod 3 )\mod 3\\
0 &= ( \beta_0+ \beta_1 + \beta_2 + ... + \beta_i )\mod 3
\end{align*}
We are using the properties of modular arithmetic over here. We are using the following properties:
\begin{enumerate}
\item
	$(A + B) \mod C = ((A \mod C) + (B \mod C)) \mod C$
	\\and
\item
	$(A * B) \mod C = ((A \mod C) * (B \mod C)) \mod C$
\end{enumerate}
Thus if a number is divisible by three then the sum if it's digits are also divisible by three.\\ \\
$\Leftarrow$ Let$\beta$ be a whole number such that the sum of its digits ($\beta_0+ \beta_1 + \beta_2 + ... + \beta_i $) is divisible by 3. Then we can say:
\begin{align*}
( \beta_0+ \beta_1 + \beta_2 + ... + \beta_i )\mod 3 &= 0\\
( (\beta_0) \mod 3 + (\beta_1)\mod 3+(\beta_2)\mod 3 + ... + (\beta_i)\mod 3 )\mod 3 &= \\
( ((\beta_0)\mod 3*1) \mod 3 + ((\beta_1)\mod 3 * 1) \mod 3 +((\beta_2)\mod 3 * 1)\mod 3)+&\\ \mod 3 + ... + ((\beta_i)\mod 3 * 1) \mod 3 )\mod 3 &=\\
( ((\beta_0)\mod 3*(10^0)\mod3) \mod 3 + ((\beta_1)\mod 3 * (10^1)\mod 3) \mod 3 & +\\((\beta_2)\mod 3 * (10^2))\mod 3) \mod 3 + ... + ((\beta_i)\mod 3 * (10^i)\mod 3) \mod 3 )\mod 3 &= \\
( (\beta_0*10^0) \mod 3 + (\beta_1 * 10^1) \mod 3 + (\beta_2 * 10^2) \mod 3 + ... + (\beta_i * 10^i) \mod 3 )\mod 3 &=\\
( \beta_0*10^0 + \beta_1 * 10^1 + \beta_2 * 10^2 + ... + \beta_i * 10^i ) \mod 3 &= 0\\
( \beta) \mod 3 &= 0
\end{align*}

Thus if the sum of the digits of a number are divisible by 3 then then number itself is divisible by 3.

\end{proof}

\end{document}