\documentclass[12pt, fullpage]{article}
\usepackage{amssymb,latexsym,amsmath,amscd,epsfig,amsthm,graphicx, verbatim}
\newcommand*{\QEDA}{\hfill\ensuremath{\blacksquare}}%
\pagestyle{empty}

\input epsf
\newdimen\epsfxsize

\parindent=0pt
\setlength{\evensidemargin}{0.0cm}
\setlength{\oddsidemargin}{-1.7cm}
\setlength{\topmargin}{-3.2cm}
%\setlength{\baselineskip}{20pt}
\setlength{\textwidth}{19cm}
\setlength{\textheight}{23cm}

\newcommand{\ds}{\displaystyle}
\newcommand{\un}{\underline}
\newcommand{\R}{\mathbb R}


\begin{document}

\begin{flushleft}
\textbf{Nilay Bhatt Feb. 19 2017}		
\end{flushleft}

\begin{center}
	\textbf{Divisibility of Three}
\end{center}

\begin{center}
		
{\bf MATH 450 Seminar in Proof}
 \\
\end{center}

\begin{center}
\textbf{A number is divisible by three if and only if the sum of the digits is divisible by 3.}
\end{center}

\begin{proof}
 Let $\beta$ be a number. We can then write $\beta$ as:
\begin{align*}
\beta &= \beta_0*10^0 + \beta_1 * 10^1 + \beta_2 * 10^2 + ... + \beta_i * 10^i
\end{align*}
\textbf{NOTE: }We are using the following properties of modular arithmetic:
\begin{enumerate}
\item
	$(A + B) \bmod C = ((A \bmod C) + (B \bmod C)) \bmod C$
	\\and
\item
	$(A * B) \bmod C = ((A \bmod C) * (B \bmod C)) \bmod C$
\end{enumerate}
where $A,B,C$ are some numbers.\\ \\
$\Rightarrow$ Let that $\beta$ is divisible by 3. Then $\beta\bmod 3 = 0$, thus we can write the expansion as: 
\begin{align*}
(\beta)\bmod 3 &= ( \beta_0*10^0 + \beta_1 * 10^1 + \beta_2 * 10^2 + ... + \beta_i * 10^i ) \bmod 3\\
0 &= ( \beta_0*10^0 + \beta_1 * 10^1 + \beta_2 * 10^2 + ... + \beta_i * 10^i ) \bmod 3  \\
&= [ (\beta_0*10^0) \bmod 3 + (\beta_1 * 10^1) \bmod 3 + (\beta_2 * 10^2) \bmod 3 + ... + (\beta_i * 10^i) \bmod 3 ]\bmod 3 \\
&= [ \{(\beta_0)\bmod 3*(10^0)\bmod3\} \bmod 3 + \{(\beta_1)\bmod 3 * (10^1)\bmod 3\} \bmod 3 \\& +\{(\beta_2)\bmod 3 * (10^2))\bmod 3\} \bmod 3 + ... + \{(\beta_i)\bmod 3 * (10^i)\bmod 3\} \bmod 3 ]\bmod 3 \\
&=[\{(\beta_0)\bmod 3*1\} \bmod 3 + \{(\beta_1)\bmod 3 * 1\} \bmod 3 +\{(\beta_2)\bmod 3 * 1\}\bmod 3 \\& + ... + \{(\beta_i)\bmod 3 * 1\} \bmod 3 ]\bmod 3\\
&= [ (\beta_0) \bmod 3 + (\beta_1)\bmod 3+(\beta_2)\bmod 3 + ... + (\beta_i)\bmod 3 ]\bmod 3\\
0 &= [ \beta_0+ \beta_1 + \beta_2 + ... + \beta_i ]\bmod 3
\end{align*}
Thus if a number is divisible by three then the sum if its digits is also divisible by three.\\
$\Leftarrow$ Let $\beta$ be a number such that the sum of its digits $\beta_0+ \beta_1 + \beta_2 + ... + \beta_i $ is divisible by 3. From the algebra done in the equations above, and by the definition of equality, we can follow the last equation\\ \begin{center}
$[ \beta_0+ \beta_1 + \beta_2 + ... + \beta_i ]\bmod 3 = 0$
\end{center} from bottom up and we thus we can then say that $\beta$ is also divisible by 3.
\begin{comment} 
\begin{align*}
( \beta_0+ \beta_1 + \beta_2 + ... + \beta_i )\bmod 3 &= 0\\
( (\beta_0) \bmod 3 + (\beta_1)\bmod 3+(\beta_2)\bmod 3 + ... + (\beta_i)\bmod 3 )\bmod 3 &= \\
( ((\beta_0)\bmod 3*1) \bmod 3 + ((\beta_1)\bmod 3 * 1) \bmod 3 +((\beta_2)\bmod 3 * 1)\bmod 3)+&\\ \bmod 3 + ... + ((\beta_i)\bmod 3 * 1) \bmod 3 )\bmod 3 &=\\
( ((\beta_0)\bmod 3*(10^0)\bmod3) \bmod 3 + ((\beta_1)\bmod 3 * (10^1)\bmod 3) \bmod 3 & +\\((\beta_2)\bmod 3 * (10^2))\bmod 3) \bmod 3 + ... + ((\beta_i)\bmod 3 * (10^i)\bmod 3) \bmod 3 )\bmod 3 &= \\
( (\beta_0*10^0) \bmod 3 + (\beta_1 * 10^1) \bmod 3 + (\beta_2 * 10^2) \bmod 3 + ... + (\beta_i * 10^i) \bmod 3 )\bmod 3 &=\\
( \beta_0*10^0 + \beta_1 * 10^1 + \beta_2 * 10^2 + ... + \beta_i * 10^i ) \bmod 3 &= 0\\
( \beta) \bmod 3 &= 0
\end{align*}
\end{comment}

Thus if the sum of the digits of a number are divisible by 3 then then number itself is divisible by 3.

\end{proof}

\end{document}