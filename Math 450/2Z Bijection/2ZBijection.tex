\documentclass[14pt, fullpage]{article}
\usepackage{amssymb,latexsym,amsmath,amscd,epsfig,amsthm,graphicx,verbatim, extsizes}

\newcommand*{\QEDA}{\hfill\ensuremath{\blacksquare}}%
\pagestyle{empty}

\input epsf
\newdimen\epsfxsize

\parindent=0pt
%\setlength{\evensidemargin}{-2.0cm}
%\setlength{\oddsidemargin}{1.5cm}
\setlength{\topmargin}{-1.5cm}
%\setlength{\baselineskip}{20pt}
%\setlength{\textwidth}{19cm}
%\setlength{\textheight}{23cm}

\newcommand{\ds}{\displaystyle}
\newcommand{\un}{\underline}
\newcommand{\R}{\mathbb R}
\newcommand{\Z}{\mathbb Z}


\begin{document}
\begin{center}
		
{\bf MATH 450 Seminar in Proof}
 \\
\end{center}
	Let $f: \Z \rightarrow 2\Z$ be defined by $f(x) = 2x - 6.$ Prove that $f$ is a bijection.
\begin{proof}
	Let $f$ be the function defined as in the question.\\
	\textit{One-to-One:} Let $f(x_1) = f(x_2)$, then \\
	\begin{equation}
	2x_1 - 6 = 2x_2 - 6
	\end{equation}
	\begin{equation}	
	2x_1 = 2x_2
	\end{equation}
	\begin{equation}
	x_1 = x_1 
	\end{equation}
	This  means that if $f(x_1) = f(x_2)$ then, $x_1 = x_1$ thus $f$ is one-to-one.

	\textit{Onto:} Let $y \in 2\Z$ such that, $x = \dfrac{y+6}{2}$. We will show that $x \in \Z$ \\
	\begin{equation}			
		\begin{split}
			f(x) &= 2x - 6 \\ 
				 &= 2\left(\dfrac{y+6}{2}\right) - 6\\
				 &= y+6-6\\
			f(x) &= y			
		\end{split}
	\end{equation}
	Also, since $y \in 2\Z$, and 6 is even we know that $y-6 \in 2\Z$. Furthermore, $\frac{y-6}{2}$ can be even or odd, but more importantly it will be an integer. Thus, $x \in \Z$.
	This  means that for every $y \in 2\Z$ there exists an $x \in \Z$ and making $f$ onto. Thus $f$ is bijective.
\end{proof}
\end{document}