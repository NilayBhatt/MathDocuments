\documentclass[12pt, fullpage]{article}
\usepackage{amssymb,latexsym,amsmath,amscd,epsfig,amsthm,graphicx}
\newcommand*{\QEDA}{\hfill\ensuremath{\blacksquare}}%
\pagestyle{empty}

\input epsf
\newdimen\epsfxsize

\parindent=0pt
\setlength{\evensidemargin}{0.0cm}
\setlength{\oddsidemargin}{-1.7cm}
\setlength{\topmargin}{-3.2cm}
%\setlength{\baselineskip}{20pt}
\setlength{\textwidth}{19cm}
\setlength{\textheight}{23cm}

\newcommand{\ds}{\displaystyle}
\newcommand{\un}{\underline}
\newcommand{\R}{\mathbb R}


\begin{document}
\begin{flushleft}
\textbf{Nilay Bhatt Feb. 8 2017}		
\end{flushleft}
\begin{center}
	\textbf{Results about Euler's path and circuits}
\end{center}
\begin{center}
		
{\bf MATH 450 Seminar in Proof}
 \\
\end{center}
\textbf{Definition 1.1: \textit{Graph: }}A $graph$ is a pair $G = (V,E)$. The elements of $V$ are the vertices (or nodes or points) and the elements of $E$ are edges (or lines) connecting the vertices.\\

\textbf{Note:}	The vertex on $G$ are referred to as $V(G)$ and the edges on $G$ are referred to $E(G)$. This is independent of the way we define a graph.\\

\textbf{Definition 1.2: \textit{Degree of a Vertex:}}: Let $G = (V,E)$ be a non empty graph is the number of edges ($E(G)$) attached to a vertex $v \in V(G)$ in $G$.\\

\textbf{Definition 1.3: \textit{Euler Path: }}An \textit{Euler Path} on a graph $G$ is a path that traverses each edge exactly once.\\

\textbf{Definition 1.4: \textit{Euler Circuit/Cycle: }}An Euler circuit on a graph $G$ is a Euler Path which starts and ends on the same node.\\

\textbf{Definition 1.5: \textit{Walk}}: A walk on a graph $G$ is a unique path using edges and vertex of $G$.\\

\textbf{Lemma: \textit{Nilay's Lemma (Not really):}} If a graph has a degree of at least two, then $G$ has a $cycle$.
\begin{proof}
Let $G$ be a finite graph. Let $v$ be a vertex in $G$ such that $v$ has at least two degree. Let us construct a walk starting from $v$. Let $v_1$ be an adjacent vertex to $v$, $v_2$ be an adjacent vertex to $v_1$ and so on. So the walk we create will look like $v \rightarrow v_1 \rightarrow v_2 \rightarrow ....\rightarrow v_k$. We can do this recursively all $k>1$ because of our hypothesis that each vertex has at least two degrees. Since $G$ is finite graph, the number of vertices it has is limited. Thus, while constructing our walk we will encounter a vertex $v_i$ which has already been traversed (already included in the walk). The path that was created from the first occurrence of $v_i$ to the second one is a $cycle$ from $v_i$ to $v_i$.
\end{proof}

\textbf{Results to be proven: }

\begin{enumerate}
	\item Any graph where the degree of every vertex is even iff it has an Euler circuit.
	\item If there are exactly two vertices $a$ and $b$ of odd degree, there is an Euler path on the graph.
\end{enumerate}
$Proof:$
\begin{enumerate}
	\item
 	$\Rightarrow$ Let $G$ be a graph which has Euler circuit $E$. When traversing $E$, when we come across any vertex $v$ through an edge $e_v(1)$, we know by definition there is another edge $e_v(2)$ that is connected to $v$. Thus making every vertex in $G$ at least degree two. Thus making every vertex in $G$ of even degree. \\\\
 	$\Leftarrow$ Let us proceed by induction. Let every vertex in a graph $G$ have an even degree. 
 	Assume, that there are only two nodes in $G$, each with two edges on them. Thus it is clear that you will end on the same node that you started with, thus making a Euler circuit on $G$. \\
 	Now,let $G$ be  graph of even degree with more than two nodes. Lets start with an arbitrary node $v_1$. Now since every node has an even degree, we can follow a trail from $v_1$ to the next node and repeat it for the next node we encounter. We know that this is possible because each time we enter a node $v_k$ we know there is an unused edge adjacent to it for us to use to traverse. Thus using a unique path we will come back on $v_1$ since there are even number of edges, we will always find an untapped edge to let us continue our path back to $v_1$.
 	\\\\
\begin{figure}[h!]
  \includegraphics[width=\linewidth	]{circuit.jpg}
  \caption{A Euler Circuit.}
  \label{fig:ec1}
\end{figure} 	
 	Figure 1. \ref{fig:ec1}Euler Circuit
	\item
	Let $G$ be a graph with Euler circuit. Thus every node/vertex has an even degree. now let us add one node say $b$ and add an edge to a node $a$ in the existing graph $G$. Note that before adding the edge from $b$ to $a$, $a$ in $G$ had an even degree. Now if we start drawing our path from $b$, and since it has only one edge connecting to $a$ we go to $a$ now, if we hypothetically ignore the edge connecting $a$ and $b$, the remainder of $G$ has nodes with even edges, thus making it a Euler circuit. Therefore, the trail will end at $a$ but since we have already used the edge connecting $a$ and $b$, we stop at $a$. Thus we were able to traverse all the edges in $G$ +$b$ exactly once, starting from $b$ and ending at an edge $a$ in $G$.
\end{enumerate}

\QEDA

\end{document}