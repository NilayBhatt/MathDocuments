\documentclass[12pt, fullpage]{article}
\usepackage{amssymb,latexsym,amsmath,amscd,epsfig,amsthm,graphicx,verbatim}

\newcommand*{\QEDA}{\hfill\ensuremath{\blacksquare}}%
\pagestyle{empty}

\input epsf
\newdimen\epsfxsize

\parindent=0pt
%\setlength{\evensidemargin}{-2.0cm}
%\setlength{\oddsidemargin}{1.5cm}
\setlength{\topmargin}{-1.5cm}
%\setlength{\baselineskip}{20pt}
%\setlength{\textwidth}{19cm}
%\setlength{\textheight}{23cm}

\newcommand{\ds}{\displaystyle}
\newcommand{\un}{\underline}
\newcommand{\R}{\mathbb R}
\newcommand{\Z}{\mathbb Z}
\newcommand{\N}{\mathbb N}


\begin{document}
\begin{center}
		
{\bf MATH 450 Seminar in Proof}
 \\
\end{center}
	Every integer greater than 1 is expressible as a product of primes.
\begin{proof}
	Let us proceed by contradiction. Let $n > 1$ be the least integer that cannot be expressed as a product of primes. By construction, $n$ has no divisors that are prime, so $n$ is not a prime number. Thus $n$ is composite number. Thus, there exists an integer $a$ where $1 < a < n$ such that, $a$ divides $n$. Since $a$ divides $n$ we can write $n$ as $n = ab$ where $1<b<n$. Since $n$ is the smallest number that does not have prime factors, we can say that $a = p_1 \cdot p_2 \cdot p_3 \cdot ... \cdot p_n$ and $b = q_1 \cdot q_2 \cdot q_3 \cdot ... \cdot q_n$ where $p_i, q_i$, where $i \in \N$ are prime factors of $a$ and $b$. Thus we can rewrite $n = ab$ as, $n = (p_1 \cdot p_2 \cdot p_3 \cdot ... \cdot p_n)(q_1 \cdot q_2 \cdot q_3 \cdot ... \cdot q_n)$. Now we are able to represent $n$ as a product of prime numbers. Thus $\rightarrow\leftarrow$. Therefore every integer greater than 1 is expressible as a product of primes.
\end{proof}
\end{document}