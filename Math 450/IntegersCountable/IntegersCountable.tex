\documentclass[14pt]{article}
\usepackage{amssymb,latexsym,amsmath,amscd,epsfig,amsthm,graphicx,verbatim, extsizes}

\newcommand*{\QEDA}{\hfill\ensuremath{\blacksquare}}%
\pagestyle{empty}

\input epsf
\newdimen\epsfxsize

\parindent=0pt
%\setlength{\evensidemargin}{-2.0cm}
%\setlength{\oddsidemargin}{1.5cm}
\setlength{\topmargin}{-1.5cm}
%\setlength{\baselineskip}{20pt}
%\setlength{\textwidth}{19cm}
%\setlength{\textheight}{23cm}

\newcommand{\ds}{\displaystyle}
\newcommand{\un}{\underline}
\newcommand{\R}{\mathbb R}
\newcommand{\Z}{\mathbb Z}
\newcommand{\N}{\mathbb N}


\begin{document}
\begin{center}
		
{\bf MATH 450 Seminar in Proof}
 \\
\end{center}
	Prove that the set of integers are countable.
\begin{proof}
	 We know that a set $X$ is countable if there exists a bijection in $f: \N \rightarrow X$.
	 Let $f: \N \rightarrow \Z$ b defined as \[
f(n) =
\begin{cases}
 - \left( \dfrac{n-1}{2} \right) & \text{if n is odd} \\
 \dfrac{n}{2} & \text{if n is even}
\end{cases}
\]
We know that $f(n)$ is a well defined function from the class lecture. So we will show that it is a bijection. \\ 
\textbf{One-to-One: \\}
Let $f(a) = f(b)$ where $a,b \in \N$. Therefore, either $f(a) = - \left( \dfrac{a-1}{2} \right)$ and $f(b) = - \left( \dfrac{b-1}{2} \right)$ or $f(a) = \dfrac{a}{2}$ and $f(b) = \dfrac{b}{2}$. 
Note that if we have $f(b) = - \left( \dfrac{b-1}{2} \right)$ and $f(a) = \dfrac{a}{2}$ with out loss of generality, we will have $b = 1 - a$ which makes $b \leq 0$ implies $b \notin \N$. Therefore both $f(a)$ and $f(b)$ will have the same structure.


In both the former cases we get $a=b$ if $f(a) = f(b)$. thus the function is one-to-one.\\
\textbf{Onto: \\}
Let $y \in \Z$. If $y > 0$ then we have $f(2y) = \dfrac{2y}{2} = y$. Since $y > 0$ implies $2y >0$ and $2y \in \N$. 
 Thus $2y = n \in \N$. If $y < 0$ then we have $ f(-2y + 1) = - \left( \dfrac{-2y + 1-1}{2} \right) = y$. Since $y <0$, $-2y > 0$ and thus $-2y + 1 > 0$. Thus $-2y+1 \in \N$. Hence, the function is onto.

Thus the $f:\N \rightarrow \Z$ with \[
f(n) =
\begin{cases}
 - \left( \dfrac{n-1}{2} \right) & \text{if n is odd} \\
 \dfrac{n}{2} & \text{if n is even}
\end{cases}
\] is well defined and is a bijection. Thus the set of integers are countable.
\end{proof}
\end{document}