\documentclass[14pt]{article}
\usepackage{amssymb,latexsym,amsmath,amscd,epsfig,amsthm,graphicx,verbatim, extsizes}

\newcommand*{\QEDA}{\hfill\ensuremath{\blacksquare}}%
\pagestyle{empty}

\input epsf
\newdimen\epsfxsize

\parindent=0pt
%\setlength{\evensidemargin}{-2.0cm}
%\setlength{\oddsidemargin}{1.5cm}
\setlength{\topmargin}{-1.5cm}
%\setlength{\baselineskip}{20pt}
%\setlength{\textwidth}{19cm}
%\setlength{\textheight}{23cm}

\newcommand{\ds}{\displaystyle}
\newcommand{\un}{\underline}
\newcommand{\R}{\mathbb R}
\newcommand{\Z}{\mathbb Z}
\newcommand{\N}{\mathbb N}


\begin{document}
\begin{center}
		
{\bf MATH 450 Seminar in Proof}
 \\
\end{center}
	Prove that the set of all infinite binary strings are uncountable.
\begin{proof}
	Let us proceed by contradiction. Let $A$ be the set that represents the set of all possible infinite binary strings. Then we assume that $A$ is countable, $i.e$ $|A| = | \N |$ then we can say that there is a bijection $f: \N \rightarrow A$. Since $A$ is countable we can list the elements of $A$ which are mapped from $\N$ through $f$.
	\begin{align*}
	f(1) &= a_{11} a_{12}a_{13}a_{14}... \\
	f(2) &= a_{21}a_{22}a_{23}a_{24}... \\
	f(3) &= a_{31}a_{32}a_{33}a_{34}... \\
	....\\
	....\\
	.... 		
	\end{align*}
	where $a_{ij}$ is the $j^{th}$ digit of $f(i)$ where $i,j \in \N$. Since $f$ is a bijection, $f$ is onto. Thus for all $a \in A$ there exits an $x \in \N$. Let us define digit $d_i$ as
	 \[
d_i =
\begin{cases}
 0 & \text{if } a_{ii} = 1 \\
 1 & \text{if } a_{ii} \neq 1
\end{cases}
\] 
Thus the infinite binary string $d$ whose $i^{th}$ digit is $d_i$ is different from all $f(i)$. By our assumption $d \in A$, but there is no pre-image in $\N$ such that $f(i) = d$ for any $i$. Thus there is contradiction. Thus the set of all infinite binary is not countable. 
\end{proof}
\end{document}