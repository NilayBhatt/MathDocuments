\documentclass[14pt]{article}
\usepackage{amssymb,latexsym,amsmath,amscd,epsfig,amsthm,graphicx,verbatim, extsizes}

\newcommand*{\QEDA}{\hfill\ensuremath{\blacksquare}}%
\pagestyle{empty}

\input epsf
\newdimen\epsfxsize

\parindent=0pt
%\setlength{\evensidemargin}{-2.0cm}
%\setlength{\oddsidemargin}{1.5cm}
\setlength{\topmargin}{-1.5cm}
%\setlength{\baselineskip}{20pt}
%\setlength{\textwidth}{19cm}
%\setlength{\textheight}{23cm}

\newcommand{\ds}{\displaystyle}
\newcommand{\un}{\underline}
\newcommand{\R}{\mathbb R}
\newcommand{\Z}{\mathbb Z}
\newcommand{\N}{\mathbb N}


\begin{document}
\begin{center}
		
{\bf MATH 450 Seminar in Proof}
 \\
\end{center}
	Prove by induction that $|P(A)| = 2^n$ if $|A| = n$.\\
\begin{proof}
	 By induction. \\
	
	\textbf{Base Case:} Let $A$ be a set with 0 elements. Then the $P(A) = \{ \emptyset \}$. Thus $|P(A)| = 1 = 2^0$.\\
	\textbf{Inductive Hypothesis:} Assume if $|A| = n$ then $|P(A)| = 2^n$ is true.
	\textbf{Inductive Step:}
	\\ We will prove that if $|A| = n+1$ then $|P(A)| = 2^{n+1}$. Now, let $A$ be a set such that $|A| = n+1$. Let $B = A - \{a\}$ where $a \in A$. Then $|B| = n$. Thus $|P(B)| = 2^n$ from our hypothesis.
	
	 Also we can split the subsets of $A$ into two parts, namely subsets that contain $a$ and subsets that does not $i.e$ $P(B)$. Note that $P(B)$ do not have any sets in it that contain $a$. Let $B_1, B_2, B_3,...,B_{2^n}$ be the elements of $P(B)$. Then, $B_1 \cup \{a\} \in P(A)$, $B_2 \cup \{a\} \in P(A)$, $B_3 \cup \{a\} \in P(A)$, ..., $B_{2^n} \cup \{a\} \in P(A)$ are the subsets of $A$ that contain the element $a$. Since the union of each $B_i$ and $\{a\}$ produces $2^n$ subsets and $|P(B)| =2^n$, then $|P(A)| = 2^n + 2^n = 2^n(1+1) = 2^{n+1}$. 
\end{proof}
\end{document}